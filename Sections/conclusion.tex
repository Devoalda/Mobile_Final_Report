% Conclusion
% • Conclusion of the project objectives
% • Any other possible future enhancements

\chapter{Conclusion}\label{ch:Conclusion}



\section{Future Enhancements}\label{future_enhancements}

% drafted future enhancements here tgt with rq's ocr - Jovian
In future developments, we plan to enhance the notification system by adding interactive features that enable users to mark notifications as read or bookmark them for later perusal directly within the notification interface. Such advancements are in line with our ongoing commitment to refine the app based on user feedback and preferences, prioritizing ease of use and continuous engagement to ensure that updates effectively address the needs of our community. This process of iterative improvement underscores our dedication to delivering a user experience on the "Study Together" platform that is both seamless and compelling.

Regarding the OCR functionality, we recognize that the processing time is significantly affected by the image size. Although we have implemented a cropping function to mitigate this, processing large blocks of text remains challenging. Future considerations will include additional pre-processing steps like image scaling or compression to enhance processing speed. The current system's limitations in recognizing handwritten text, due to the constraints of the TessBase API, are also acknowledged. To overcome these challenges, we are considering the exploration of alternative OCR libraries or frameworks that may offer improved accuracy. For particularly complex or specialized text recognition tasks, developing a custom OCR model presents itself as a promising solution, offering the potential to significantly enhance the platform's capability to accurately process a wider range of textual inputs.

Another potential enhancement with regards to elevating the display of math formulas and associated content within math question answers is to integrate with a technology like MathView, a library designed for Android applications, which supports LaTeX syntax for formulas. By integrating MathView, users would enjoy a more seamless and visually appealing experience when viewing and interacting with math content. Moreover, incorporating features such as the ability to zoom in or out on formulas, highlight specific parts of a formula, or animate the steps of solving a math problem could further enrich the user experience, making the app more engaging for students studying math related questions.

% Looking forward, we are exploring potential enhancements for the notification system, such as introducing interactive elements that allow users to mark notifications as read or save them for later review directly from the notification interface. These proposed improvements reflect our commitment to evolving the app in line with our users' feedback and preferences, emphasizing simplicity in design and the necessity for continual user engagement to ensure that new features effectively meet user needs. This ongoing iteration process highlights our dedication to creating a seamless, engaging user experience on the "Study Together" platform

%future work for OCR if needed%
% The processing time of the OCR is greatly influenced by the size of the image. Although a cropper function is implemented to potentially address such issue, recognising huge block of text is still a challenging task. Additional pre-processings such as image scaling or compressing can be considered to improve the processing time. The function also lacks on the ability to accurately recognise hand-written text, constrained by the capability of the TessBase API. To mitigate the limitations encountered in text recognition, especially with hand-written text, exploring alternative OCR libraries or frameworks could be beneficial. Additionally, for highly specialized or challenging tasks, training a custom OCR model could be a viable option.
