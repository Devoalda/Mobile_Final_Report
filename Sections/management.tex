% Project Management & Planning
% • Summary information on the team’s product backlog & sprint 1, 2 & 3
% backlogs
% • Description of how the team arrived at the product backlog or any other
% planning process
% • Brief description or summary of the product and sprint backlogs
% • Description of how the team arrived at the sprint backlog or any other
% planning process
% • Excerpts or main user stories and tasks in the main report, rest in appendix
% • Entire product backlog (table of tasks) in appendix
% • Each sprint backlog (table of tasks) in appendix
% • Review of each sprint’s progress, review and retrospective
% • Team workflow and a description of the team’s branching model and merging
% model


\chapter{Project Management}\label{ch:project_management}

\section{Summary of Backlogs and Sprints}
% Read and mod if funny (Ben) %
% Backlog Summary & description of how the team arrive at this backlog + planning process


The team arrived at the product backlog through a thorough planning process that involved identifying and prioritizing key features based on user needs and project goals. Initially, the team conducted user surveys to understand the target audience's preferences and pain points. This information was then used to define the core functionalities required for the application.

Following this, the team engaged in collaborative discussions and brainstorming sessions to refine the initial list of features and prioritize them based on their importance and feasibility. The team also discussed about technical considerations, such as the integration of advanced technologies like OCR and AI-generated answer features, to ensure that the application is capable to deliver a satisfying user experience.

Through this iterative planning process, the team created a comprehensive product backlog that outlined the key features and functionalities required for the development of the "Study Together" application. This backlog serves as a roadmap, to guide the team in the implementation of the application's features and ensuring that the final product meets the needs and expectations of our users.

The product backlog encompasses various aspects, from core functionalities like posting questions and receiving responses to essential user management features such as sign-in functionality, account creation and profile management. Additionally, it includes features aimed at enhancing user engagement and experience, such as notification systems, real-time chat, user following, and an explore page for curated question collections. Moreover, the backlog highlights the integration of advanced technologies like Optical Character Recognition (OCR) and AI-generated answer features to augment the application's capabilities.

\subsection{Sprint 1 Summary}
% Sprint 1 Summary
% looks good - Jovian
The sprint backlog for Sprint 1 details the specific tasks and activities planned for the implementation during this iteration. Tasks include setting up Firebase integration, implementing sign-in functionality, designing user registration forms, and configuring password reset capabilities. Furthermore, user profile management features, such as profile editing and management functionalities, are slated for implementation. Additionally, tasks related to footer navigation layout setup and Firebase setup are prioritized to ensure a seamless user interface and easy to use navigation.

\subsection{Sprint 2 Summary}
% Sprint 2 Summary
% Added more stuff - Jovian
% In Sprint 2, the development focuses on core functionalities crucial for collaborative learning. This sprint entails implementing the question-asking features for timely responses, efficient search capabilities for quick access to resources, and real-time chat functionality to facilitate seamless communication among users. Additionally, advanced AI-driven answers using WolframAlpha are integrated to enhance the platform's utility. To further elevate user engagement and enrich the learning experience on the platform, a basic notification function has also been introduced. This feature is designed to keep users up-to-date by providing daily updates on new posts and activities, ensuring members of the community do not miss out on valuable discussions and resources. Through these combined efforts, Sprint 2 aims to create a more interactive and comprehensive learning environment, fostering both individual and collaborative growth.

% remodded again and removed real-time chat in sprint 2 - Jovian
In Sprint 2, the "Study Together" team focused on adding key functionalities essential for creating a conducive environment for collaborative learning. This phase of development saw the introduction of features that allowed users to pose questions and receive prompt responses, as well as enhanced search capabilities for swift access to educational resources. To augment the platform's utility, WolframAlpha was integrated for providing advanced, AI-driven answers, offering a richer learning experience.

A major enhancement during this sprint was the development of a basic notification function. Designed to elevate user engagement, this feature aims to keep the community informed about daily new posts and activities. The implementation of this notification system ensures that users stay updated with the latest discussions and resources, fostering an environment where valuable educational exchanges are consistently highlighted and accessible.

\subsection{Sprint 3 Summary}
% Sprint 3 Summary
% looks good ah
% Sprint 3, our focus shifts to enhancing user experience and platform aesthetics. This sprint includes the implementation of a curated questions page, which offers users a centralized hub to explore a diverse list of questions. Furthermore, the integration of Optical Character Recognition (OCR) technology streamlines the process of digitizing printed text, providing users with a seamless question creation experience. Additionally, the sprint entails refining the platform's visual appeal through the implementation of a dynamic color scheme and theme. Through these tasks, Sprint 3 aims to elevate user satisfaction and usability, further solidifying "Study Together" as a premier educational platform. As the development process comes to the end, we integrates all the indivisual parts for a complete system testing. To facilitate this, a continuous integration is implemented via Github Workflows. This approach allows multiple contributors to test their own components' compatibility with system.

% remodded again and addded more stuff, real-time chat shifted to here as well - Jovian
In Sprint 3, the "Study Together" team dedicated efforts to significantly enhancing the user experience and the visual aesthetics of the platform. A notable feature introduced in this sprint is the real-time chat functionality, designed to facilitate instant communication among users. This addition aims to foster a more interactive community by enabling live discussions, peer support, and collaborative learning opportunities. The chat feature is expected to complement the platform's educational resources by providing a space for real-time exchange of ideas and clarifications on various topics.

Alongside the chat implementation, we focused on further enhancing the notification system introduced in Sprint 2. Building upon the basic notification functionality, enhancements include more personalized notifications that cater to the users' interests and recent interactions on the platform. These improvements are designed to keep users engaged and informed about relevant content, encouraging more active participation in the learning community.

Additionally, Sprint 3 saw the deployment of a curated questions page, offering users a tailored list of questions aimed at piquing their interest and encouraging deeper exploration of subjects. The integration of Optical Character Recognition (OCR) technology marked another leap forward, simplifying the process of converting printed text into digital format, thereby easing the submission of questions from various sources. The sprint also focused on elevating the platform's visual design through dynamic color schemes and themes, enhancing the overall user interface and experience.

As we approach the culmination of the development process, Sprint 3 involved integrating all individual components for comprehensive system testing. This phase was supported by the implementation of continuous integration through GitHub Workflows, enabling seamless collaboration among multiple contributors. This strategy ensured that each component was compatible with the system as a whole, paving the way for a smooth and efficient testing phase. Through these concerted efforts, Sprint 3 contributed significantly to making "Study Together" a more engaging, aesthetically pleasing, and user-friendly educational platform, ready to meet the diverse needs of its growing community.

\section{Sprint Backlog Planning Process}
% Read and mod if funny (Ben) %
% Description of how the team arrived at the sprint backlog or any other planning process %

% remodded to this, can mod if too long etc. - Jovian %
Our path to creating the sprint backlog for "Study Together" was a deliberate and cooperative effort, underlining our dedication to crafting an application that is both innovative and centered around the user experience. Initially, we embarked on this journey with comprehensive user surveys, aiming to gain deep insights into the preferences, challenges, and expectations of our target audience. These insights became the foundation for our feature selection, ensuring our development aligned with user needs from the very beginning.

Following the surveys, our team participated in extensive brainstorming sessions, utilizing the feedback received to not just enumerate possible features but to visualize the entire user journey on "Study Together." This phase involved critical analysis of each feature's potential to boost collaborative learning and user interaction, ensuring we were adding true value to our platform.

As we moved forward, our focus shifted to the prioritization of these features, assessing them based on their relevance to the user experience and their technical feasibility. This step was crucial, as it required us to weigh our aspirations against the practicalities of our resources and technical capabilities. Concurrently, we explored the integration of advanced technologies such as Optical Character Recognition (OCR) and AI-driven solutions to not only meet but exceed user expectations, setting our platform apart in the realm of digital education.

The culmination of these efforts was the formulation of our final sprint backlog through an iterative refinement process. This backlog, a meticulously curated list of tasks and features, was directly aligned with our strategic goals and user requirements. It was dynamic, allowing for continuous adjustments and updates to reflect our agile development approach.

Sprint planning meetings played a pivotal role in this phase, serving as essential gatherings where tasks were delegated, schedules were established, and resources were allocated. These meetings ensured team alignment and clarity on the sprint objectives, fostering a collaborative and ownership-driven atmosphere.

This sprint backlog planning process became the backbone of our project management strategy, ensuring that each sprint was purposeful, manageable, and in harmony with our broader aim of enhancing the "Study Together" experience. It allowed us to navigate challenges and capitalize on new opportunities, all while remaining committed to our goal of enriching the collaborative learning environment.

By adhering to this structured yet adaptable planning process, we not only maintained momentum in our development efforts but also ensured that "Study Together" continues to evolve in a manner that resonates with our users' evolving needs and incorporates the forefront of technological advancements.

% Read and mod if funny (Ben) %
% Review of each sprint's progress, review and retrospective %
\section{Review of Each Sprint's Progress, Review, and Retrospective}

Each sprint in the development of "Study Together" has been a step forward in realizing our vision for a collaborative learning platform. Here is a summary of the progress made in each sprint, along with a review and retrospective of our efforts:

% commented all this out coz abit repetitive - Jovian %
% \subsection{Sprint 1}
% Sprint 1 focused on laying the foundation for the application. We successfully set up Firebase integration, implemented sign-in functionality, designed user registration forms, and configured password reset capabilities. User profile management features were also developed, including editing and management functionalities. Tasks related to footer navigation layout setup and Firebase setup were prioritized to ensure a seamless user interface and easy navigation.

% \subsection{Sprint 2}
% In Sprint 2, our focus was on implementing core functionalities crucial for collaborative learning. We implemented question-asking features for timely responses, efficient search capabilities for quick access to resources, and real-time chat functionality to facilitate seamless communication among users. Additionally, we integrated advanced AI-driven answers using WolframAlpha to enhance the platform's utility. Sprint 2 aimed to elevate user engagement and enrich the learning experience on the "Study Together" platform.

% \subsection{Sprint 3}
% Sprint 3 shifted the focus to enhancing user experience and platform aesthetics. We implemented a curated questions page, offering users a centralized hub to explore a diverse array of questions. The integration of Optical Character Recognition (OCR) technology streamlined the process of digitizing printed or handwritten text, providing users with a seamless question creation experience. Additionally, we refined the platform's visual appeal through the implementation of a dynamic color scheme and theme. Sprint 3 aimed to elevate user satisfaction and usability, further solidifying "Study Together" as a premier educational platform.

% \subsection{Review and Retrospective}
% Throughout each sprint, we have learned valuable lessons and made continuous improvements to our development process. Regular review meetings helped us assess our progress, identify any bottlenecks or challenges, and make necessary adjustments. Retrospectives at the end of each sprint allowed us to reflect on what went well and what could be improved, helping us iterate and refine our approach for future sprints.


% remodded to this and splitted the info like this instead - Jovian %
% be nice if you all talk more about the challenges you all faced more indepth here %
\subsection{Sprint 1 Review and Retrospective}\label{sprint1_review_retrospective}
The initial sprint was pivotal in establishing the foundational infrastructure for the "Study Together" platform. The team succeeded in integrating backend services with Firebase, and implementing key user functionalities such as sign-in, user registration, and password reset features. Early efforts in user profile management laid the groundwork for more sophisticated user interaction capabilities in future sprints. Despite these successes, the sprint was not without its challenges, particularly in smoothing out the user profile management feature which encountered several unexpected technical hurdles. This experience underscored the importance of robust testing procedures. The team learned valuable lessons about the necessity of incorporating comprehensive testing early in the development process to identify and resolve issues more efficiently, ensuring a smoother implementation of features.


% remodded to this and splitted the info like this instead - Jovian %
% be nice if you all talk more about the challenges you all faced more indepth here %
% Haven't talk abt implementing chat in this sprint, need to go into refining alr? - removed alr - Jovian
\subsection{Sprint 2 Review and Retrospective}\label{sprint2_review_retrospective}
% In Sprint 2, our team was dedicated to enriching the "Study Together" platform with key functionalities essential for fostering an environment of collaborative learning. This phase saw the rollout of several important features, such as the capability for users to pose questions, improved search functionalities, real-time chat, and the integration of AI-driven responses through WolframAlpha. A pivotal enhancement was the establishment of a basic notification system, a step forward in increasing user interaction with the platform.

% Among the advancements, the implementation of the notification system stood out as particularly challenging. The complexity of developing this feature in Kotlin, a language that had not been extensively covered in our coursework up to that point, posed a significant hurdle. The intricacies of creating a service for notifications in Kotlin required the team to engage in self-directed learning and problem-solving, venturing beyond the classroom teachings to grasp the necessary concepts and techniques.


% Moreover, implementing the real-time chat feature proved to be a substantial task, as it demanded a delicate balance between sophisticated functionality and maintaining an intuitive user experience. These hurdles underscored the value of an iterative development approach and the utility of incorporating user feedback early and often in the development process. Such challenges brought to light the essential nature of continuous testing and adjustments, reinforcing the team's commitment to producing user-centric features that adhere to the lofty expectations for the platform. This sprint, with its blend of achievements and learning opportunities, emphasized the importance of flexibility and resilience in the face of technical challenges, guiding our approach to feature development moving forward.

% Configuring the secrets property file to securely store the Gemini API key as an environment variable in the project's build config presented a significant challenge. Initially, the team encountered security concerns when attempting to store the key directly, prompting exploration of alternative secure storage methods. However, due to a lack of online documentation providing clear guidance on this specific task, the team had to rely on the closest available sample tutorial and conduct their own experiments to achieve a successful implementation.

% be nice if you all talk more about the challenges you all faced more indepth here % - Jovian
% add on here ah benben %
In Sprint 2, the "Study Together" development team focused on adding essential features to nurture a collaborative learning environment. During this phase, enhancements included the ability for users to pose questions and improved search functionalities. A significant achievement was the initial rollout of a basic notification system, aimed at boosting user engagement by keeping them informed about new questions and updates. This implementation marked a crucial step towards fostering a more interactive and connected user experience on the platform.

One of the most notable challenges encountered was the development of the notification system using Kotlin—a programming language that, at the time, had not been extensively explored in our coursework. The team faced the complexities of crafting a notification service in Kotlin head-on, embarking on a journey of self-directed learning to understand the intricacies involved. This endeavor required us to venture beyond traditional classroom knowledge, applying real-world problem-solving skills to overcome the technical hurdles presented by the notification feature's development.

Additionally, configuring the secrets property file for secure storage of the Gemini API key posed another significant challenge. The team grappled with security concerns related to storing the key directly in the codebase. With limited guidance available in online documentation, we turned to the most relevant tutorials and engaged in experimental problem-solving to devise a secure method for integrating the API key into the project's build configuration. This process highlighted the importance of security in software development and underscored the necessity of innovative thinking when faced with documentation gaps.

These experiences in Sprint 2 underscored the value of an iterative development approach, emphasizing the critical role of user feedback, continuous testing, and adjustments in the development process. The hurdles we overcame not only taught us valuable technical skills but also reinforced our commitment to creating user-centric features that meet the high expectations for "Study Together." This phase of development was a testament to the team's flexibility and resilience, laying a solid foundation for the future enhancement of the platform.

% remodded to this and splitted the info like this instead - Jovian %
% be nice if you all talk more about the challenges you all faced more indepth here %
\subsection{Sprint 3 Review and Retrospective}\label{sprint3_review_retrospective}
% In Sprint 3, the team's efforts were concentrated on elevating the user experience and aesthetic appeal of the "Study Together" platform. This sprint saw the successful implementation of a curated questions page, integration of Optical Character Recognition (OCR) technology, and the introduction of dynamic color schemes to enhance the platform's visual design. These developments significantly improved the platform's usability and visual appeal, making it more engaging for users. However, the team faced challenges in achieving the right balance between feature complexity and maintaining an intuitive user interface, especially for the curated questions feature. The experience reinforced the value of user-centered design and the impact of visual aesthetics on user satisfaction and engagement. Lessons learned from this sprint include the importance of simplicity in design and the need for ongoing user feedback to ensure that new features meet user needs effectively.

% Sprint 3 marked a significant phase in the development of "Study Together," with a concentrated effort on refining the notification system and enhancing the overall user experience and aesthetic appeal of the platform. Building on the groundwork of Sprint 2, we introduced a visual upgrade in the notification system by adding the poster's profile picture to the previously simple display of the poster's username and question title. This personalization helps users quickly recognize who is posting and maintains the interface's clean and accessible design by excluding the question description, which was previously considered content-heavy.

% Additionally, we redefined the notification screen to solely feature daily new posts, setting it apart from the comprehensive home screen feed. This adjustment aims to streamline content discovery, allowing users to effortlessly distinguish between daily updates and general browsing, thus enhancing app navigability and functionality.

% Beyond notifications, Sprint 3 was pivotal in advancing the "Study Together" platform through the successful implementation of a curated questions page, the integration of Optical Character Recognition (OCR) technology, and the debut of dynamic color schemes. These enhancements significantly boosted the app's visual design and usability, engaging users more deeply with the content.

% Despite these advancements, we encountered challenges in striking an optimal balance between the complexity of new features and maintaining an intuitive user interface, particularly with the curated questions feature. This experience underscored the importance of user-centered design and the profound effect of visual aesthetics on user engagement and satisfaction.

% Looking forward, we are exploring potential enhancements for the notification system, such as introducing interactive elements that allow users to mark notifications as read or save them for later review directly from the notification interface. These proposed improvements reflect our commitment to evolving the app in line with our users' feedback and preferences, emphasizing simplicity in design and the necessity for continual user engagement to ensure that new features effectively meet user needs. This ongoing iteration process highlights our dedication to creating a seamless, engaging user experience on the "Study Together" platform.

Sprint 3 was a transformative stage in the evolution of "Study Together," focusing on refining the notification system and enhancing the platform's user experience and visual appeal. This sprint built upon the foundation set in Sprint 2, introducing a significant update to the notification system by incorporating the poster's profile picture. This enhancement aids in quick user recognition of the poster, streamlining the interface by excluding the previously deemed content-heavy question descriptions.

The notification screen was also restructured to exclusively showcase daily new posts, differentiating it from the comprehensive array of updates found on the home screen feed. This change was designed to simplify the process of discovering new content, making it easier for users to differentiate between daily notifications and regular content browsing, thereby improving the app's navigability and user engagement.

Additionally, Sprint 3 brought several key developments to "Study Together," including the launch of a curated questions page, the integration of Optical Character Recognition (OCR) technology, and the introduction of dynamic color schemes. These updates significantly enhanced the visual design and functionality of the app, creating a more immersive and user-friendly experience.

However, integrating these new features while maintaining an intuitive user interface presented challenges, especially concerning the curated questions feature. These hurdles underscored the importance of user-centered design and the impact of aesthetics on user engagement and satisfaction, reinforcing the team's commitment to creating a more immersive and user-friendly learning environment.

% remodded to this and splitted the info like this instead - Jovian %
% be nice if you all talk more about the challenges you all faced more indepth here %
\subsection{Sprint Review and Retrospective Summary}\label{sprint_review_retrospective_summary}
Each sprint brought its own set of achievements and challenges, driving home the importance of a flexible, user-focused development process. From establishing the foundational infrastructure to enhancing user engagement and experience, the team navigated through various technical and design challenges, learning valuable lessons along the way. These iterative review and retrospective processes have been instrumental in refining the "Study Together" platform, ensuring that each sprint moves us closer to delivering a comprehensive and user-friendly learning environment. As we continue with subsequent sprints, these insights will guide our development efforts, ensuring that we remain responsive to user needs and committed to excellence in our pursuit of creating an engaging and effective collaborative learning platform.


% Read and mod if funny (Ben) %
% looks good, helped you to add the title alr %
% Team workflow and a description of the team's branching model and merging model %
\subsection{Team Workflow}\label{team_workflow}
Our team follows an agile workflow, utilizing the Scrum framework for project management. We conduct regular sprint planning meetings to prioritize tasks and set goals for each sprint. Daily stand-up meetings keep everyone in sync and allow us to discuss any blockers or issues that arise. At the end of each sprint, we hold sprint review meetings to demo completed work and gather feedback, followed by sprint retrospective meetings to reflect on what went well and what could be improved.

\subsection{Branching Model and Merging Model}\label{bm_mm}
For version control, we use Git and follow a branching model based on GitFlow. We have a main branch that reflects the production-ready code. For each new feature or bug fix, we create a feature branch from the main branch. Once the feature is completed, we merge the feature branch back into the main branch through a pull request. Code reviews are an integral part of our process, ensuring that all code changes meet our quality standards before being merged into the main branch.

Our merging model follows a strict protocol to maintain code integrity and stability. All pull requests must pass automated tests and receive approval from at least one other team member before being merged into our master branch.
