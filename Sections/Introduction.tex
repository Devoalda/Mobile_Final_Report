% Introduction
% • Description of the motivations and goals of the project
% • Intended problem to be solved and how it corresponds to the theme
\chapter{Introduction}\label{introduction}

\section{Problem Statement}\label{problem_statement}
The project addresses the challenge faced by new students at SIT, who encounter a significant academic transition. This transition brings forth obstacles such as adapting to a new teaching style and managing increased workload, impacting the learning experience of freshmen. The goal is to design a supportive educational system that aids the students in navigating university life, fostering a smoother transition.

\section{Implications}\label{implications}
Without proactive support, students may experience increased academic stress, reduced satisfaction, and ultimately, higher dropout rates. Institutions must invest in tailored support systems to facilitate adaptation to higher education demands.

\section{Persona}\label{persona}
The project's target persona, Alex, is a first-year university student at SIT. He seeks to attain quality education and establish connections while overcoming concerns about academic workload and social integration. The persona is described in Section \ref{persona}

\subsection{Motivation}
The motivation behind this project is to alleviate the academic transition challenges faced by new university students, particularly those at SIT. By creating a supportive online community, we aim to empower students like Alex to voice their questions and collaborate with their peers. This platform seeks to facilitate a smoother adaptation to university life, addressing concerns about academic workload and social integration.

\subsection{Goals}
The primary goal of this project is to enhance the academic experience of new students at SIT by providing them with a supportive educational system. This system aims to help students adapt to the university's learning style and manage the increased workload more effectively. Additionally, the project seeks to reduce academic stress, improve student satisfaction, encourage teamwork, and lower dropout rates by offering tailored support that addresses the unique challenges faced by freshmen.

\section{Solution}\label{solution}
The proposed solution is a Peer-to-Peer Learning Platform inspired by platforms like Reddit and Instagram. It facilitates direct peer-to-peer interaction, question prioritization through curated question feeds, user exploration, user interactions, and effective profile management.



% Old
% \chapter{Introduction}\label{introduction}

% \section{Problem Statement}\label{problem_statement}
% The challenge at hand involves new students at SIT facing a \textbf{significant academic leap} from their previous educational experiences. The transition to university life brings forth various obstacles, including \textit{adapting to a different teaching style} and \textit{managing a more demanding workload}. This situation is a common struggle among \textbf{freshmen}, impacting their overall learning experience. The goal is to understand and address these challenges effectively by designing supportive educational systems and strategies that aid students in \textit{navigating the complexities of university life}, fostering a \textit{smoother transition}, and enhancing their learning journey.

% \section{Implications}\label{implications}
% Without proactive measures to support students in their transition to university life, there is a risk of \textbf{increased academic stress}, \textit{reduced satisfaction with the learning experience}, and potentially \textit{higher dropout rates}. Recognizing and addressing these challenges is crucial for the overall success and well-being of students, emphasizing the need for institutions to invest in \textit{tailored support systems}, \textit{mentorship programs}, and \textit{teaching methodologies} that facilitate a more seamless adaptation to the demands of higher education.
 
% \subsection{Motivation}

% This project is driven by a commitment to easing the academic transition for new university students. Recognizing the common challenges faced by freshmen, we aim to create a supportive online community where students can learn and grow together. This platform empowers introverted students to voice their questions and fosters a collaborative environment where peers can assist each other during this crucial phase.



% \section{Solution}\label{solution}

% Building upon the identified challenges, we propose a multifaceted solution designed to elevate the university experience.

% \subsection{A Peer-to-Peer Learning Platform}

% Inspired by popular platforms like Reddit and Instagram, our application fosters a forum-like environment where students, like Alex, can:

% \begin{itemize}
%   \item Ask questions and seek answers directly from peers.
%   \item Upvote and prioritize valuable responses through a refined voting system.
%   \item Browse a curated feed of questions tailored to their interests.
%   \item Follow users and explore a wider range of inquiries through a dedicated "Explore" page.
%   \item Manage their profiles and interactions effectively.
% \end{itemize}










